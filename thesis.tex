% !TeX spellcheck = en_GB
%%%%%%%%%%%%%%%%%%%%%%%%%%%%%%%%%%%%%%%%%%
%                                        %
%    Engineer thesis LaTeX template      %
%                                        %
%%%%%%%%%%%%%%%%%%%%%%%%%%%%%%%%%%%%%%%%%%



\documentclass[a4paper,twoside,12pt]{book}
\usepackage[utf8]{inputenc}
\usepackage[T1]{fontenc}
\usepackage{amsmath,amsfonts,amssymb,amsthm}
\usepackage[polish,british]{babel}
\usepackage{indentfirst}
\usepackage{lmodern}
\usepackage{graphicx}
\usepackage{hyperref}
\usepackage{booktabs}
\usepackage{tikz}
\usepackage{pgfplots}
\usepackage{mathtools}
\usepackage{geometry}
\usepackage[page]{appendix}

\usepackage{setspace}
\onehalfspacing


\frenchspacing

\usepackage{listings}
\lstset{
	language={},
	basicstyle=\ttfamily,
	keywordstyle=\lst@ifdisplaystyle\color{blue}\fi,
	commentstyle=\color{gray}
}

%%%%%%%%%



%%%%%%%%%%%% FANCY HEADERS %%%%%%%%%%%%%%%

\usepackage{fancyhdr}
\pagestyle{fancy}
\fancyhf{}
\fancyhead[LO]{\nouppercase{\it\rightmark}}
\fancyhead[RE]{\nouppercase{\it\leftmark}}
\fancyhead[LE,RO]{\it\thepage}


\fancypagestyle{onlyPageNumbers}{%
   \fancyhf{}
   \fancyhead[LE,RO]{\it\thepage}
}

\fancypagestyle{PageNumbersChapterTitles}{%
   \fancyhf{}
   \fancyhead[LO]{\nouppercase{\it\rightmark}}
   \fancyhead[RE]{\nouppercase{\it\leftmark}}
   \fancyhead[LE,RO]{\it\thepage}
}


%%%%%%%%%%%%%%%%%%%%%%%%%%%
% listings
\usepackage{listings}
\lstset{%
language=C++,%
commentstyle=\textit,%
identifierstyle=\textsf,%
keywordstyle=\sffamily\bfseries, %\texttt, %
%captionpos=b,%
tabsize=3,%
frame=lines,%
numbers=left,%
numberstyle=\tiny,%
numbersep=5pt,%
breaklines=true,%
morekeywords={descriptor_gaussian,descriptor,partition,fcm_possibilistic,dataset,my_exception,exception,std,vector},%
escapeinside={@*}{*@},%
%texcl=true, % wylacza tryb verbatim w komentarzach jednolinijkowych
}
%%%%%%%%%%%%%%%%%%%%%%%%%%%%%%%%%%%%

%%%% TODO LIST GENERATOR %%%%%%%%%

\usepackage{color}
\definecolor{brickred}      {cmyk}{0   , 0.89, 0.94, 0.28}

\makeatletter \newcommand \kslistofremarks{\section*{Remarks} \@starttoc{rks}}
  \newcommand\l@uwagas[2]
    {\par\noindent \textbf{#2:} %\parbox{10cm}
{#1}\par} \makeatother


\newcommand{\remark}[1]{%
{%\marginpar{\textdbend}
{\color{brickred}{[#1]}}}%
\addcontentsline{rks}{uwagas}{\protect{#1}}%
}

%%%%%%%%%%%%%% END OF TODO LIST GENERATOR %%%%%%%%%%%

% some issues...

\newcounter{PagesWithoutNumbers}

\newcommand{\hcancel}[1]{%
    \tikz[baseline=(tocancel.base)]{
        \node[inner sep=0pt,outer sep=0pt] (tocancel) {#1};
        \draw[red] (tocancel.south west) -- (tocancel.north east);
    }%
}%

\newcommand{\MonthName}{%
  \ifcase\the\month
  \or January% 1
  \or February% 2
  \or March% 3
  \or April% 4
  \or May% 5
  \or June% 6
  \or July% 7
  \or August% 8
  \or September% 9
  \or October% 10
  \or November% 11
  \or December% 12
  \fi}


%%%%%%%%%%%%%%%%%%%%%%%%%%%%%%%%%%%%%%%%%%%%%%
% Helvetica font macros for the title page:
\newcommand{\headerfont}{\fontfamily{phv}\fontsize{18}{18}\bfseries\scshape\selectfont}
\newcommand{\titlefont}{\fontfamily{phv}\fontsize{18}{18}\selectfont}
\newcommand{\otherfont}{\fontfamily{phv}\fontsize{14}{14}\selectfont}

%%%%%%%%%%%%%%%%%%%%%%%%%%%%%%%%%%%%%%%%%%%%%%
%%%%%%%%%%%%%%%%%%%%%%%%%%%%%%%%%%%%%%%%%%%%%%
%%%%%%%%%%%%%%%%%%%%%%%%%%%%%%%%%%%%%%%%%%%%%%
%%%%%%%%%%%%%%%%%%%%%%%%%%%%%%%%%%%%%%%%%%%%%%
%%%%%%%%%%%%%%%%%%%%%%%%%%%%%%%%%%%%%%%%%%%%%%
%%%%%%%%%%%%%%%%%%%%%%%%%%%%%%%%%%%%%%%%%%%%%%
%%%%%%%%%%%%%%%%%%%%%%%%%%%%%%%%%%%%%%%%%%%%%%


\newcommand{\Author}{Damian Kucharski}
\newcommand{\Supervisor}{Łukasz Wróbel, PhD}
\newcommand{\Title}{Automated Gradient Boosting Framework for classification data}
\newcommand{\Polsl}{Silesian University of Technology}
\newcommand{\Faculty}{Faculty of Automatic Control, Electronics and Computer Science}


\begin{document}
	
%%%%%%%%%%%%%%%%%%  Title page %%%%%%%%%%%%%%%%%%%
\pagestyle{empty}
{
	\newgeometry{top=2.5cm,%
	             bottom=2.5cm,%
	             left=3cm,
	             right=2.5cm}
	\sffamily
	\rule{0cm}{0cm}
	
	\begin{center}
	\includegraphics[width=29mm]{polsl}
	\end{center}
	\vspace{1cm}
	\begin{center}
	\headerfont \Polsl
	\end{center}
	\begin{center}
	\headerfont \Faculty
	\end{center}
	\vfill
	\begin{center}
	\titlefont Engineer  thesis
	\end{center}
	\vfill
	
	\begin{center}
	\otherfont \Title\par
	\end{center}
	
	\vfill
	
	\vfill
	
	\noindent\vbox
	{
		\hbox{\otherfont author: \Author}
		\vspace{12pt}
		\hbox{\otherfont supervisor: \Supervisor}
	}
	\vfill

   \begin{center}
   \otherfont Gliwice,  \MonthName\ \the\year
   \end{center}	
	\restoregeometry
}


\cleardoublepage


\rmfamily
\normalfont


%%%%%%%%%%%% statements required by law and Dean's office %%%%%%%%%%
\cleardoublepage

\begin{flushright}
załącznik nr 2 do zarz. nr 97/08/09
\end{flushright}

\vfill

\begin{center}
\Large\bfseries Oświadczenie
\end{center}

\vfill

Wyrażam  zgodę / Nie wyrażam zgody*  na  udostępnienie  mojej  pracy  dyplomowej / rozprawy doktorskiej*.

\vfill

Gliwice, dnia {\selectlanguage{polish}\today}

\vfill

\rule{0.5\textwidth}{0cm}\dotfill

\rule{0.5\textwidth}{0cm}
\begin{minipage}{0.45\textwidth}
{\begin{center}(podpis)\end{center}}
\end{minipage}

\vfill

\rule{0.5\textwidth}{0cm}\dotfill

\rule{0.5\textwidth}{0cm}
\begin{minipage}{0.45\textwidth}
{\begin{center}\rule{0mm}{5mm}(poświadczenie wiarygodności podpisu przez Dziekanat)\end{center}}
\end{minipage}


\vfill

* podkreślić właściwe




%%%%%%%%%%%%%%%%%%%%%
\cleardoublepage

\rule{1cm}{0cm}

\vfill

\begin{center}
\Large\bfseries Oświadczenie promotora
\end{center}

\vfill

Oświadczam, że praca „\Title” spełnia wymagania formalne pracy dyplomowej inżynierskiej.

\vfill



\vfill

Gliwice, dnia {\selectlanguage{polish}\today}

\rule{0.5\textwidth}{0cm}\dotfill

\rule{0.5\textwidth}{0cm}
\begin{minipage}{0.45\textwidth}
{\begin{center}(podpis promotora)\end{center}}
\end{minipage}

\vfill



\cleardoublepage


%%%%%%%%%%%%%%%%%% Table of contents %%%%%%%%%%%%%%%%%%%%%%
\pagenumbering{Roman}
\pagestyle{onlyPageNumbers}
\tableofcontents

%%%%%%%%%%%%%%%%%%%%%%%%%%%%%%%%%%%%%%%%%%%%%%%%%%%%%
\setcounter{PagesWithoutNumbers}{\value{page}}
\mainmatter
\pagestyle{PageNumbersChapterTitles}

%%%%%%%%%%%%%% body of the thesis %%%%%%%%%%%%%%%%%


\chapter{Introduction}

\begin{itemize}
\item introduction into the problem domain
\item settling of the problem in the domain
\item objective of the thesis
\item scope of the thesis
\item short description of chapters
\item clear description of contribution of the thesis's author – in case of more authors table with enumeration of contribution of authors
\newline
------------------------------
\newline
The roots of artificial intelligence and machine learning appear quite early in history, one of earliest examples could be introduction of Bayes Theorem at 1700s, followed by regression analysis at 1800s. The field constantly evolved, but it got much more attention pretty recently when computational possibilities expanded enough for big data to be processed and more complex algorithms (like neural networks) to be used.
The thing that at first was called inferential statistics, data mining and recently data science and machine learning has application in almost any field imaginable. Tools such as spam filters, chatbots, web search results, analysis of medical imaging, reconstruction of images, text generation and autocompletion, robotics, video and board games, logistic processes optimization and much more - they all benefit from intelligent solutions that are built into them.
Machines are able to exceed performance of humans in many critical fields, well trained machine learning model can analyse MRI imaging as well as radiologist with 20 years of experience. Very recently, at November 2020, company called OpenAI, known for creation of AlphaZero - best chess playing program and AlphaGo - first ever program to beat human in the ancient chineese game of Go, has created program called AlphaFold2. It solves task of so called protein folding, a 50-year-old grand challenge in biology, with tremendous 90 percent accuracy, exceeding performance of its first version from 2 years before by over 30 percent. This work by some is named the biggest accomplishment of artificial intelligence studies and is predicted to be crucial part to next great discoveries in medical and biology fields. Some applications, such as the evolutionary analysis of proteins, are set to flourish because the tsunami of available genomic data might now be reliably translated into structures. This can for example help to prevent such events like covid-19 pandemic and help to find cure for many diseases whose mechanisms are not yet well understood.
\newline
Occurances like that show two things. First - that the field of machine learning and artificial intelligence has and will have great impact on the world, and second - that the advancements in this field are very far from slowing down, in fact before mentioned AlphaFold and other novel machine learning models like transformers that pushed natural language modeling to the whole new level are the best evidence that the opposite is happening - this growth is speeding up.
\newline
However with new ideas, growth of the availability of data and processing power, complexity of the models expanded and because of that it is now much harder to make them perform optimally. It is much harder to break into field now, than it was just few years ago. Ideas ones considered state-of-art are now mostly used only for educational purposes and even for experienced specialists it is much harder to create model that actually achieves it's full potential. The quality of final model of the data depends mostly on two factors: 
\item The quality and prepossessing of the data, feature engineering
\item Choosing right model and its hyperparameters.

Because of the reason of growing complexity of the modeling task the idea of automatic solution arose and is mostly referred to as Auto Modeling, Auto Machine Learning or just Auto-ML. Such a solution should automate at least one part of modeling pipeline, it means that it should either preprocess data for modeling purposes, select right model and optimize it's hyperparameter choices, or to do both.
Such a solution gives good baseline and push to the right direction, saving time otherwise spent for experiments that are destined to fail. In short words, Auto-ML is the idea to incorporate machine learning to create machine learning models. It is still rather new field, however it already gives promising results and likely in short amount of time it will completely replace big chunk of traditional work performed by data scientists and other practitioner in the field, just as spreadsheets replaced need for manual calculation on the paper for most of the use cases.
\newline
There are many approaches to solving such a problem, especially in terms of optimizing for model choice. 
Usually the hardest part of the task it to optimize hyperparameters. Most of them are different for each model type and while their optimal choice is crucial for good modeling of given data they can't be directly learned from it. Let's analyze simple example of probabilistic model that decides whether a person will be accepted to college given results of maths and physics test. Example of hyperparameter for such a model could be for example the way it measures its error which is often named cost function, or objective function. It cannot be optimized during model training as it is by itself used to optimize the model. It has to be selected beforehand and is constant for whole optimization procedure.
Another example of hyper parameter is not directly model specific but necessary for the model to learn - namely optimization method. What is important is that this method by itself can have it's own hyperparameters. More details on different hyper parameters will be explored later in the work but for now what is important is that the more complex the model is, the more hyper parameters which usually means much harder procedure of finding their optimal values.
\newline
The objective of this thesis is development of Auto-ML functionality in form of python library. It covers both data processing and model selection. Models built by it are based on gradient boosted trees algorithms that are optimized using users method of choice, implemented inside library. Namely, grid search and Bayesian optimization are supported. Apart from that, web based application was developed that can be accessed using internet or run on ones own machine. It allows for training model and viewing visualizations summarizing training and explaining models decisions. As with growing complexity of modern machine learning algorithms it is increasingly hard to understand their decisions and it is very common to just treat them as black box. However good understanding of model inner workings has proved in the past to be crucial in further enhancement of its performance. Good example can be paper "Visualizing and Understanding Convolutional Networks" (Trzeba to jakoś ładnie zacytować)[https://arxiv.org/abs/1311.2901] by Matthew D Zeiler et. al. This study analyzed winning convolutional neural network architecture of previous ImageNet competition. In result the next iteration of this competition has been won by this team of researchers.
Because of that software prepared to accompany this work has been designed to offer tools that make it easy to understand how model actually works to  make further research and domain knowledge application as simple as possible.
The library also offers lower level interface, that gives possibility to quickly ensemble machine learning pipelines for programmers without sacrificing complexity of it. Because of that it is much simpler to perform repetitive, standard tasks in data processing and modeling, at the same time enhancing simplicity and readability of the code. Hyperparametrs of such models can be then optimized with before mentioned methods.
The project was co-created with Arkadiusz Czerwiński who also ensured that regression tasks are well covered, whilst my scope was to prepare library for handling classification problems. Most of the work however was done in collaboration as big chunk of whole code was needed to handle both of use cases.


\end{itemize}

\chapter{[Problem analysis]}

\begin{itemize}
\item  problem analysis
\item state of the art, problem statement
\item  literature research (all sources in the thesis have to be referenced \cite{bib:article,bib:book,bib:conference,bib:internet})
\item description of existing solutions (also scientific ones, if the problem is scientifically researched), algorithms,  location of the thesis in the scientific domain
\end{itemize}

As was mentioned in the introduction, Auto-ML library that is subject of this thesis has to do both things - preprocess the data and optimize hyperparameters. Created model should perform classification tasks with high efficiency, measured using metrics such as accuracy and f1-score.
After training process it should be possible to plot information concerning effectiveness of training and model characteristics explanation.
All of these problems will be now explained so the solution presented by this thesis will be more understandable.

\section{Classification problem explanation}

The main problem that the solution should solve is so called classification task. Such task consists of predicting fixed class label for given set of data. Classification is most the frequent problem in machine learning use cases and can be performed on any kind of input data. There are two types of classification problems - binary classification and multilabel classification.
In the first case model, given the data, has to choose one of only two classes, usually representing True/False values or presence/absence of something. Examples include classifying e-mails as spam and not-spam or predicting whether some person has some medical condition or not.
In the second case there are at least 3 labels, usually predefined, and model has to choose one of them that most likely categorizes data it was presented. Depending on the algorithm used to perform the task, it can be either split into multiple binary classification problems using so-called one-vs-all approach in which for each class individual problem of classifying if given data point belongs to given class of not. However some algorithms allow to perform multilabel classification at one step. Example of such problems is for example famous MNIST classification task, in which given picture of handwritten digit, program has to output what digit is present in the image.
To illustrate such a problem simple example will be provided. The problem to be solved will be classification of student college acceptance given scores of two tests. Data is presented on the following graph.


\begin{figure}[h]
    \centering
    \includegraphics[scale=0.75]{college.png}
    \caption{Example of classification task}
    \label{fig:mesh1}
\end{figure}


\subsection{Bias and Variance}

There are two important concepts that are necessary to be known to understand what the whole model optimization, model selection, data processing and others are about. They are called \emph{Bias} and \emph{Variance} and in a way they measure quality of the model.
The model is said to have high bias if it is too simple to reproduce the relationships that are present in the data. Such a model is inherently incapable of representing them and therefore is highly biased. It is also often said that the model \emph{under-fits} the data. On the other side there are model that have high variance. This situation occurs when model is too complex and in a way "tries to hard" to model relationships it sees in the data. Such a model may learn to make perfect predictions for the data it has seen but it would most likely completely fail to do it with the new data. It therefore has high variance because the relationships it models highly vary depending on the specific set of data points that was presented to it during training process. Such a model is often said to \emph{over-fit} the data.
Ideally the model should have low bias and low variance but it is often impossible because when one decreases another increases. The task then is to find good middle-ground and the problem just referenced is often named \emph{Bias-Variance Tradeoff}

\begin{figure}[h]
    \centering
    \includegraphics[scale=1.5]{overfitting_2.png}
    \caption{Example of underfitting and overfitting}
    \label{fig:mesh1}
\end{figure}

https://www.geeksforgeeks.org/underfitting-and-overfitting-in-machine-learning/

\subsection{Classifiers, tree based methods and gradient boosting}

There are many machine learning models that are suitable use for classification tasks. First there are simple ones like logistic regression. This model tries to fit the line, or in case of problem located in higher dimensions - hyperplane, that seperates classes. Then there are more complicated models like support vector machines. These were state-of-art for long time until deep learning happened. Such a model tries to find a boundary that guarantees the biggest margin, in linear algebra sense, between classes. With the use of so called kernel trick and functions known as kernels it can achieve impressive results. Then recently the deep learning era emerged and models known as neural networks became the the best in many application, they however usually require big amounts of data to achieve good results.
In this thesis however tree-based models were chosen to be the base of the solution. They have some qualities that make them especially good choice for tackling the problem and some of these qualities as well as some theory of how the models work will be shortly explained.

\subsubsection{Decision Tree and Gradient Boosting}

Decision tree is a simple model trying to learn simple decision rules inferred from the training data. Such rules can be easily visualized as a series of if-else statements. These behavior can be achieved with different methods however what each of them tries to do is finding optimal splits - so decisions - that explain differences between samples. The sequence of such rules create a tree structure that can be easily interpreted.


\begin{figure}[h]
    \centering
    \includegraphics[scale=0.4]{decision_tree.png}
    \caption{Visualization of decision tree}
    \source{Source: https://cutt.ly/hh7jmin}
    \label{fig:mesh1}
\end{figure}

To achieve better performance of the machine learning model there is a possibility to use a method called ensembling. Ensembling means training multiple machine learning models and then combining their predictions to get final result. This approach often gives better results than using these models separately. 
One of the approaches that are based on the idea of ensembling is gradient boosting.
It works by creating many so-called weak learners and sequentially training them based on the results they give. Most often used components of such a method are small decisions trees. Such a trees are constrained on their maximum depth - so in other words the number of decisions (or splits) they can make. 
Because of that none of them can fit the data well but also it's impossible to overfit with such a model.
Then after creating the tree it's error is measured and knowing that error and particular samples that model got especially wrong - next tree tries to act on this error and correct it. This way every single model tries to "boost" aggregate complexity of the ensemble. Such a process is repeated n times, where n is a hyperparameter. For every dataset the optimal number of trees in ensemble may be different and therefor should be optimized by experiments or other methods that will be described later and that are the core of the framework.  

\begin{figure}[h]
    \centering
    \includegraphics[scale=0.75]{boosting.png}
    \caption{Boosting algorithm}
    \source{Source: https://cutt.ly/xh7vqXf}
    \label{fig:mesh1}
\end{figure}

\subsection{Existing implementations}

\subsubsection{XGBoost}

XGBoost is the most known library providing gradient boosting solutions. It was firstly created by Tianqi Chen and then continued by other creators. It is available for most popular operating systems - Windows, Linux and MacOS. It's implementations can be seen in many programming languages, commonly used in field of data science and machine learning. Namely Python, C++, Julia, Java, Perl and Scala support XGBoost. 
The things that make it different from other frameworks are  for example

\begin{itemize}
\item Clever penalization of trees
\item A proportional shrinking of leaf nodes
\item Newton Boosting
\item Extra randomization parameter
\item Implementation on single, distributed systems and out-of-core computation
\end{itemize}


\subsubsection{LightGBM}

Second framework is called LighXBM and as the name suggests it's main goal is to be fast and lightweight. Thanks to that it can be used on not so powerful machines that do not posses great computing powers and big volumes of memory. While firstly created b Guolin Ke it was later incorporated as a Microsofts project. It's approach to growing the decisions trees is also different that of most of its competitors - growing trees leaf-wise instead of tree-level-wise. It assumes that this approach usually yield the largest decrease in loss. Thanks to all these traits this library is very versatile and can be used almost anywhere and also it is great for quick prototyping and experimenting. 

\subsubsection{CatBoost}

Another framework is CatBoost. It is newer than the former as it was first launched at the middle of the year 2017. While not that popular as other solutions it is certainly very interesting and powerful. It is successfully used in such applications as recommender systems, personal assistants, self-driving cars, weather prediction and others. What is most remarkable in it is that it support GPU computing which greatly enhances the speed of training and also enables possibility for creating more complex models on more complex data. It also has built in visualization tools that make it easy to understand some insights that can be drawn from the training process and also to make tuning the algorithm easier. It also natively supports some preprocessing features like encoding categorical variables and missing values.

\subsection{Why gradient boosting?}

The choice of gradient boosting instead of other popular classification methods is not an accident. It has the most important traits for the auto-ml solution. It has proven to be effective many times. Kaggle platform which is a place where machine learning specialists can participate in data science and machine learning competitions has a long record of gradient boosting and especially XGBoost being a primary tool that made winners winners. 
It being tree-based also helps in model explainability which helps in turn optimization procedure. The fact that the model can then be interpreted by human more easily makes Auto-ML solutions more suitable for further improvements as it is easier to understand models drawbacks and fix them.


\section{Data preprocessing}

Before classification model can be trained, often the necessary step is to firstly preprocess it. Term preprocessing means performing transformations of the data such that machine learning model of our choice can actually interpret and learn from it. Some of the most important will be now discused.

\subsection{Missing value treatment}

Datasets are often incomplete. There are many reasons for that. Depending on the nature of the data, or the process that generated it, the cause for that may be different. There are many ways this problem can be addressed and different approaches are suitable in different situations.
There are three main ways missing values can be handled. First one is just throwing away part of the data that is missing. It can be done in one of the two ways. First is often used when some feature has mostly missing values. In such a situation one can just delete it entirely from dataset. The other situation is often practised when there is not many of missing values of some feature. In such a situation the entries with undefined value for the feature can be deleted. This approach however deletes some potentially important information given by other features.
Another way to treat missing values is to impute them. Imputing means setting them to some values. Most popular approaches include for example setting them to most frequent value for the feature, or if the values are numerical - to mean or median. This approaches however can increase the bias in the data, especially if percentage of missing values is high. 
The are also some more complicated and smarter ways of imputing values like Iterative Imputation, KNN Imputation or SMOTE[DODAĆ LINKI]. However their mechanisms of work will are out of scope of this thesis.
What is important however is that each of these methods, simple or not, have its advantages and disadvantages and choosing right one is more often matter of experimenting rather than anything else.
The last method that will be discussed here is missing values encoding. This is the easiest method and it also is very effective in many cases. If the feature having missing values is nominal, or in other words, categorical what can be done is introducing new class indicating that value is missing. In such a scenario if there is binary "sick" feature in the dataset the category of "not known" can be added thus making the feature taking three possible values instead of two. If the data is numerical missing values can be set to zero and additional binary feature called indicator variable can be added to dataset, telling if there was a missing value in given entry for the feature or not.

\subsection{Feature encoding}

Another necessary step for most of the cases is so called feature encoding. While humans are capable of processing words computers are not. Because of that every feature in our dataset that is represented in for of the text has to be encoded somehow - which basically means represented by numbers.
There are many ways to do it - again the correct method depends on the data and it's nature. 
In this thesis only some of them with limited explanations will be covered as the topic is very broad.
The most natural way is to just assign a fixed number for each different value the feature can take. For example if the "sick" variable could take values "YES" and "NO" then they could just be assigned to "1" and "0" respectively. This approach however very often is wrong. The problem arises when the number of possible values the feature can take is bigger than two. Encoding them in such a way in this case often could mean creating artifical order that does not exist but model would interpret the data in the way that it does exist. For example if there would be "color" feauture in the data taking values "RED", "GREEN" and "BLUE" encoding them to "0", "1" and "2" would create false relationship of BLUE being somehow more different, or "bigger" in respect to "RED" than "GREEN" is.
In such a case most often used method is so called one-hot encoding. 
The method consists of replacing given feature with n indicator variables for each possible value of the feature. What is important however that in case of such encoding the problem known as perfect multicollinearity [https://en.wikipedia.org/wiki/Multicollinearity] arises. It can be easly solved however by just discarding one of the created indicator variables. This approach potentially can greatly increase number of features in dataset and potentially cause other problems.
Last method that will be covered here is target encoding where the values are set to average of the target value for given category. This approach is easy to implement and often gives good results but increase the risk of overfitting to the training data.

\subsection{Outlier treatment}

The last important topic that needs to be addressed is outlier treatment. It is often the case that the dataset will contain entries that differ significantly from other observations. Such a big difference can happen in features, target or both. It is often hard to tell what is the cause of its existence, it may be error of measurement or some actual important cause. What is important is that some models are very sensitive to outliers and tree-based methods are one of them. Trying to account for outliers can reduce performance of the model on majority of observations. Therefore outliers need special treatment.
As with missing values we can delete them or impute them if it seems reasonable. This however deletes the information the can be carried by outlier. If it is assumed that outlier may be natural it can also be treated seperately as whole different problem, by new model that can then be combined with the model for "usual" observations to achieve better results.
There are many methods for determining whether an observation is outlier or not, most of them are based on statistical methods like computing z-scores of observations. 
Finding and treating outliers correctly can be big part of the model working well or bad and thus it should never be omitted, especially in cases when machine learning model is sensitive to them.

\begin{figure}[h]
    \centering
    \includegraphics[scale=1]{outlier.jpg}
    \caption{Linear regression is a model sensitive to outliers.}
    \label{fig:mesh1}
\end{figure}


https://www.theanalysisfactor.com/outliers-to-drop-or-not-to-drop/


\subsection{Other data processing steps}

There are also other things that can be done with data to increase model performance. Some algorithms benefit from scaling and transforming numerical values so that they fall in specific range or follow specific distribution. Sometimes new variables can be created on the base of others. This process named feature engineering requires some creativity and often also domain knowledge. Sometimes artificially creating more observations can help, this process is called data augmentation. It is hard if possible at all to name all things that can be tried and that potentially can help. Experimenting and getting experience is good chunk of knowing what to do, but the before mentioned steps are most necessary to make model work at all and are the challenge that the solution implemented as the subject of this thesis is concentrated to solve.


\section{Model optimization}

After preprocessing procedure model can be trained. Most of the algorithms do have some default settings. Ones used in this thesis are not exception. However as was stated many times before the process of model optimization doesn't end in optimization of its parameters. Before the model is trained it has to be set up to work in specific way and this settings called hyperparameters also have to be optimized. Gradient boosting models while powerful are rather difficult to optimize in terms of their settings. That's why most people just stay with default parameters or try and fail with changing them.
That's why the auto-ml solution that is being proposed in this thesis heavily focuses on automating this process.

\subsection{Cross-Validation}

To tune model hyper parameters some way to measure how well model is trained has to be set up. What is important in creating good predictive model is to make it perform well not only on the data it is being trained on but also on the data it is yet to be used on.
Because of that different methods for testing models performance were proposed. The one of the most used that is also the method chosen here is called cross-validation.
It consists of splitting training dataset to k folds. Parameter has to be specified by the user but in practice 5 fold are mainly used. Each fold consists of 1/k of the datasets entries. Then the set is splitted to train and test sets k times. For each iteration one of the k folds is chosen to be test set and all other together make training set. 
The model is then trained on the training set and its performance is measured on the testing set using specified metric.
There are different metrics that can be used to determine the quality of the classification model, most commonly accuracy is used as a metric which is basically the fraction of all examples that were classified correctly.
Sometimes the dataset is highly imbalanced and most of the observations fall to one class, while others appear rather rarely. In such a case most commonly used metrics include balanced accuracy score and F1 score.
WHen the training and testing process is performed k times the results are averaged and the performance is noted.
Knowing that performance it is then possible to try different approaches to data processing or hyperparameter choice. Repeatedly experimenting with these different approaches and measuring quality of the result using cross-validation is a good way to build great machine learning models.

\subsection{Grid and Randomized Search}

To optimize hyperparameters of the model one can choose different approaches. The most commonly used are naive approaches like Grid Search and Randomized Search.
These approaches are very similar and at first require the user to specify all values that they want to be tested. That's the first drawback of the method as it is easy to miss some possibly good options. It also enforces user to at least partially know what values may be a good fit.

\subsection{Bayesian Optimization}



\section{Existing Auto-ML tools}


\chapter{Requirements and tools}

\begin{itemize}
\item functional and nonfunctional requirements
\item use cases (UML diagrams)
\item description of tools
\item methodology of design and implementation
\end{itemize}


\chapter{External specification}
\begin{itemize}
\item hardware and software requirements
\item installation procedure
\item activation procedure
\item types of users
\item user manual
\item system administration
\item security issues
\item example of usage
\item working scenarios (with screenshots or output files)
\end{itemize}

\begin{figure}
\centering
\begin{tikzpicture}
\begin{axis}[
    y tick label style={
        /pgf/number format/.cd,
            fixed,
            fixed zerofill, % 1.0 instead of 1
            precision=1,
        /tikz/.cd
    },
    x tick label style={
        /pgf/number format/.cd,
            fixed,
            fixed zerofill,
            precision=2,
        /tikz/.cd
    }
]
\addplot [domain=0.0:0.1] {rnd};
\end{axis}
\end{tikzpicture}
\caption{A caption of a figure is \textbf{below} it.}
\label{fig:2}
\end{figure}


\chapter{Internal specification}

\begin{itemize}
\item concept of the system
\item system architecture
\item description of data structures (and data bases)
\item components, modules, libraries, resume of important classes (if used)
\item resume of important algorithms (if used)
\item details of implementation of selected parts
\item applied design patterns
\item UML diagrams
\end{itemize}


Use special environment for inline code, eg \lstinline|descriptor| or \lstinline|descriptor_gaussian|.
Longer parts of code put in the figure environment, eg. code in Fig. \ref{fig:pseudokod}. Very long listings–move to an appendix.

\begin{figure}
\centering
\begin{lstlisting}
class descriptor_gaussian : virtual public descriptor
{
   protected:
      /** core of the gaussian fuzzy set */
      double _mean;
      /** fuzzyfication of the gaussian fuzzy set */
      double _stddev;

   public:
      /** @param mean core of the set
          @param stddev standard deviation */
      descriptor_gaussian (double mean, double stddev);
      descriptor_gaussian (const descriptor_gaussian & w);
      virtual ~descriptor_gaussian();
      virtual descriptor * clone () const;

      /** The method elaborates membership to the gaussian fuzzy set. */
      virtual double getMembership (double x) const;

};
\end{lstlisting}
\caption{The \lstinline|descriptor_gaussian| class.}
\label{fig:pseudokod}
\end{figure}


\chapter{Verification and validation}
\begin{itemize}
\item testing paradigm (eg V model)
\item test cases, testing scope (full / partial)
\item detected and fixed bugs
\item results of experiments (optional)
\end{itemize}




\chapter{Conclusions}
\begin{itemize}
\item achieved results with regard to objectives of the thesis and requirements
\item path of further development (eg functional extension …)
\item encountered difficulties and problems
\end{itemize}


\begin{table}
\centering
\caption{A caption of a table is \textbf{above} it.}
\label{id:tab:wyniki}
\begin{tabular}{rrrrrrrr}
\toprule
	         &                                     \multicolumn{7}{c}{method}                                      \\
	         \cmidrule{2-8}
	         &         &         &        \multicolumn{3}{c}{alg. 3}        & \multicolumn{2}{c}{alg. 4, $\gamma = 2$} \\
	         \cmidrule(r){4-6}\cmidrule(r){7-8}
	$\zeta$ &     alg. 1 &   alg. 2 & $\alpha= 1.5$ & $\alpha= 2$ & $\alpha= 3$ &   $\beta = 0.1$  &   $\beta = -0.1$ \\
\midrule
	       0 &  8.3250 & 1.45305 &       7.5791 &    14.8517 &    20.0028 & 1.16396 &                       1.1365 \\
	       5 &  0.6111 & 2.27126 &       6.9952 &    13.8560 &    18.6064 & 1.18659 &                       1.1630 \\
	      10 & 11.6126 & 2.69218 &       6.2520 &    12.5202 &    16.8278 & 1.23180 &                       1.2045 \\
	      15 &  0.5665 & 2.95046 &       5.7753 &    11.4588 &    15.4837 & 1.25131 &                       1.2614 \\
	      20 & 15.8728 & 3.07225 &       5.3071 &    10.3935 &    13.8738 & 1.25307 &                       1.2217 \\
	      25 &  0.9791 & 3.19034 &       5.4575 &     9.9533 &    13.0721 & 1.27104 &                       1.2640 \\
	      30 &  2.0228 & 3.27474 &       5.7461 &     9.7164 &    12.2637 & 1.33404 &                       1.3209 \\
	      35 & 13.4210 & 3.36086 &       6.6735 &    10.0442 &    12.0270 & 1.35385 &                       1.3059 \\
	      40 & 13.2226 & 3.36420 &       7.7248 &    10.4495 &    12.0379 & 1.34919 &                       1.2768 \\
	      45 & 12.8445 & 3.47436 &       8.5539 &    10.8552 &    12.2773 & 1.42303 &                       1.4362 \\
	      50 & 12.9245 & 3.58228 &       9.2702 &    11.2183 &    12.3990 & 1.40922 &                       1.3724 \\
\bottomrule
\end{tabular}
\end{table}






%%%%%%%%%%%%%%%%%%%%%%%%%%%%%%%%%%%%%%%%%%
\backmatter
\pagenumbering{Roman}
\stepcounter{PagesWithoutNumbers}
\setcounter{page}{\value{PagesWithoutNumbers}}

\pagestyle{onlyPageNumbers}

%%%%%%%%%%% bibliography %%%%%%%%%%%%
\bibliographystyle{plain}
\bibliography{bibliography}

%%%%%%%%%  appendices %%%%%%%%%%%%%%%%%%%

\begin{appendices}




\chapter*{List of abbreviations and symbols}

\begin{itemize}
\item[DNA] deoxyribonucleic acid
\item[MVC] model--view--controller
\item[$N$] cardinality of data set
\item[$\mu$] membership function of a fuzzy set
\item[$\mathbb{E}$] set of edges of a graph
\item[$\mathcal{L}$] Laplace transformation
\end{itemize}


\chapter*{Listings}

(Put long listings in the appendix.)

\begin{lstlisting}
partition fcm_possibilistic::doPartition
                             (const dataset & ds)
{
   try
   {
      if (_nClusters < 1)
         throw std::string ("unknown number of clusters");
      if (_nIterations < 1 and _epsilon < 0)
         throw std::string ("You should set a maximal number of iteration or minimal difference -- epsilon.");
      if (_nIterations > 0 and _epsilon > 0)
         throw std::string ("Both number of iterations and minimal epsilon set -- you should set either number of iterations or minimal epsilon.");

      auto mX = ds.getMatrix();
      std::size_t nAttr = ds.getNumberOfAttributes();
      std::size_t nX    = ds.getNumberOfData();
      std::vector<std::vector<double>> mV;
      mU = std::vector<std::vector<double>> (_nClusters);
      for (auto & u : mU)
         u = std::vector<double> (nX);
      randomise(mU);
      normaliseByColumns(mU);
      calculateEtas(_nClusters, nX, ds);
      if (_nIterations > 0)
      {
         for (int iter = 0; iter < _nIterations; iter++)
         {
            mV = calculateClusterCentres(mU, mX);
            mU = modifyPartitionMatrix (mV, mX);
         }
      }
      else if (_epsilon > 0)
      {
         double frob;
         do
         {
            mV = calculateClusterCentres(mU, mX);
            auto mUnew = modifyPartitionMatrix (mV, mX);

            frob = Frobenius_norm_of_difference (mU, mUnew);
            mU = mUnew;
         } while (frob > _epsilon);
      }
      mV = calculateClusterCentres(mU, mX);
      std::vector<std::vector<double>> mS = calculateClusterFuzzification(mU, mV, mX);

      partition part;
      for (int c = 0; c < _nClusters; c++)
      {
         cluster cl;
         for (std::size_t a = 0; a < nAttr; a++)
         {
            descriptor_gaussian d (mV[c][a], mS[c][a]);
            cl.addDescriptor(d);
         }
         part.addCluster(cl);
      }
      return part;
   }
   catch (my_exception & ex)
   {
      throw my_exception (__FILE__, __FUNCTION__, __LINE__, ex.what());
   }
   catch (std::exception & ex)
   {
      throw my_exceptionn (__FILE__, __FUNCTION__, __LINE__, ex.what());
   }
   catch (std::string & ex)
   {
      throw my_exception (__FILE__, __FUNCTION__, __LINE__, ex);
   }
   catch (...)
   {
      throw my_exception (__FILE__, __FUNCTION__, __LINE__, "unknown expection");
   }
}
\end{lstlisting}

\chapter*{Contents of attached CD}

The thesis is accompanied by a CD containing:
\begin{itemize}
\item thesis (\LaTeX\ source files and final \texttt{pdf} file),
\item source code of the application,
\item test data.
\end{itemize}


\listoffigures
\listoftables
	
\end{appendices}


\end{document}


%% Finis coronat opus. 
